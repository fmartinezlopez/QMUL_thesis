\chapter*{Abstract}
\label{C:Abstract}
\addcontentsline{toc}{section}{\nameref{C:Abstract}}

% no more than 300 words
The Deep Underground Neutrino Experiment (DUNE) is a next-generation long-baseline neutrino oscillation experiment. Its primary goal is the determination of the neutrino mass hierarchy and the CP-violating phase. The DUNE physics programme also includes the detection of astrophysical neutrinos and the search for beyond the Standard Model phenomena. DUNE will consist of a near detector complex placed at Fermilab, and a modular Liquid Argon Time Projection Chamber (LArTPC) far detector to be built in the Sanford Underground Research Facility (SURF), approximately 1300 km away from the neutrino production point. The detectors will be exposed to a wide-band neutrino beam generated by a 1.2 MW proton beam, with a planned upgrade to 2.4 MW.

DUNE will be built following a staged approach with two main phases. While the Phase I ND complex is sufficient for early physics goals, a Phase II upgrade is planned in order to reach the designed sensitivity for the neutrino oscillation physics. The upgraded Phase II ND will feature ND-GAr, a magnetised high-pressure gaseous argon TPC surrounded by an electromagnetic calorimeter (ECal) and a muon tagger. The gaseous argon provides low detection thresholds, which allow for detailed measurements of nuclear effects at the interaction vertex. Additionally, the magnetic field and the ECal would enable efficient particle identification and momentum and charge reconstruction.