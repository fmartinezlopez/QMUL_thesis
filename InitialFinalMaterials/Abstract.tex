\chapter*{Abstract}
\label{C:Abstract}
\addcontentsline{toc}{section}{\nameref{C:Abstract}}

% no more than 300 words
The Deep Underground Neutrino Experiment (DUNE) is a next-generation long-baseline neutrino oscillation experiment. Its primary goal is the determination of the neutrino mass hierarchy and the CP-violating phase. The DUNE physics programme also includes the detection of astrophysical neutrinos and the search for beyond the Standard Model (BSM) phenomena. DUNE will consist of a near detector (ND) complex placed at Fermilab, and a modular Liquid Argon Time Projection Chamber (LArTPC) far detector (FD) to be built in the Sanford Underground Research Facility (SURF), approximately 1300 km away from the neutrino production point.

This thesis describes three different projects within DUNE. First, a novel strategy to improve the triggering capabilities of the DUNE FD is proposed. It uses matched filters to enhance the production of online hits across all charge collection planes. Next, the possibility of detecting neutrinos coming from dark matter (DM) annihilations in the Sun with the FD is explored. The complementarity of DUNE to this kind of DM searches is shown. Finally, the simulation and reconstruction framework of ND-GAr, the gas argon ND proposed for Phase II of DUNE, is presented. A number of additions to this are described, particularly focused on the development of the particle identification (PID) capabilities of the detector. These are then used to perform the first event selection studies with an end-to-end simulation in ND-GAr, in particular the selection of pion exclusive samples in $\nu_{\mu}$ CC interactions. All three of these projects share the common goal of enhancing the physics programme of DUNE.