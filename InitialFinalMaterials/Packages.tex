% languages, fonts and geometry
%------------------------------------------------------------------------------------
\usepackage[utf8]{inputenc}
\usepackage[italian, main = english]{babel} 			% man­ages ty­po­graph­i­cal rules and hy­phen­ation for the option languages
\usepackage[T1]{fontenc}							% manages font encoding (accents, copy and paste, ...)
%\usepackage{accents}


%%-----------------------------------------------------------------------%%
%% from QM Guidance Notes on the Submission:
%% margins at the binding edge must be not less than 40 mm (1.5 inches)
%% and other margins not less than 20 mm (.75 inches). Double or one-and-a-half
%% spacing should be used, except for indented quotations or footnotes where single
%% spacing may be used. 
%%-----------------------------------------------------------------------%%


\usepackage[ 
		layoutoffset=0pt,						% No offset on the margins of the page
		bottom=2.5cm, 
		top=5cm,	
		left=4cm, 
		right=2.5cm,			
		bindingoffset=0pt,						% We already set a bigger binding edge margin
		headheight=1.5cm				% top space dedicated to the header
		]{geometry}	
					
\usepackage{microtype, fancyhdr}	 % typesetting adjustments
\usepackage{setspace}					 %allows local control over line spacing
\usepackage{pdflscape} 					 % makes landscape pages display as landscape when opening the pdf


%maths, physics, ...
%------------------------------------------------------------------------------------
\usepackage{amsfonts,amsmath,amsthm,amssymb,amscd}	 %maths formatting
\usepackage{physics}
\usepackage[sicmds,freestanding]{hepunits}
\usepackage[compat=1.0.0]{tikz-feynman}
\usepackage[version=4]{mhchem}
\numberwithin{equation}{chapter}	
%\numberwithin{table}{chapter} 
%\numberwithin{figure}{chapter}
%\usepackage{faktor}
\allowdisplaybreaks

%tools
%------------------------------------------------------------------------------------
\usepackage{url, enumitem} 			% for linkable urls and nice lists
\usepackage[noadjust]{cite}				% biblio management
\usepackage{epigraph}						% self explanatory
\usepackage[nottoc]{tocbibind}			% to make the bibliography appear in the ToC
\usepackage{blindtext}						% this is just to generate dummy text, remove 
\usepackage{imakeidx}					% to make an index, see end of document
\usepackage{comment}         % to comment blocks of text
\makeindex


%tables & pictures
%------------------------------------------------------------------------------------
\usepackage{graphicx} 				% basic to include images and use colours
\graphicspath{{Pictures/}}				% set the path to the folder with pictures
\usepackage{xcolor}
\usepackage{longtable} 				%helps format tables that are over a page long
\usepackage{lscape} 					%allows (tables) landscape formatting
\usepackage{booktabs} 				% general table formatting
\usepackage{multicol} 					%allows table cells to cover two columns
\usepackage{multirow}		
\usepackage[labelfont=bf]{caption} 				% more tools to customise for captions
%\captionsetup[longtable]{position=below}	
\usepackage{wrapfig} 					% allows to wrap text around figures
\usepackage{subcaption}

\usepackage{array}
\newcommand{\PreserveBackslash}[1]{\let\temp=\\#1\let\\=\temp}
\newcolumntype{C}[1]{>{\PreserveBackslash\centering}p{#1}}
\newcolumntype{R}[1]{>{\PreserveBackslash\raggedleft}p{#1}}
\newcolumntype{L}[1]{>{\PreserveBackslash\raggedright}p{#1}}


% Header and footer settings, for options: https://www.ctan.org/pkg/fancyhdr
%-------------------------------------------------------------------------------------
\pagestyle{fancy}                  			 % Sets fancy header and footer
%\fancyfoot{}                            		    	 % Delete current footer settings

\renewcommand{\chaptermark}[1]{        
  \markboth{\chaptername\ \thechapter.\ #1}{}} % Lower Case Chapter marker style
\renewcommand{\sectionmark}[1]{       
  \markright{\thesection.\ #1}}				 % Lower case Section marker style
%\fancyhead[LE]{\bfseries\thepage}		 % Page number (boldface) in  Left on Even pages (if document is twoside)
%\fancyhead[RO]{\bfseries\thepage}        %  and Right on Odd pages
\fancyhead[LO]{}        %  and Right on Odd pages
\fancyhead[RO]{\bfseries\rightmark}      % Section in the Left on Odd pages
\fancyhead[RE]{}     % Chapter in the Right on Even pages (if document is twoside)
\fancyhead[LE]{\bfseries\leftmark}     % Chapter in the Right on Even pages (if document is twoside)

\let\headruleORIG\headrule
\renewcommand{\headrule}{\color{black} \headruleORIG}
\renewcommand{\headrulewidth}{0.0pt}
\usepackage{colortbl}
\arrayrulecolor{black}

\fancypagestyle{plain}{
  \fancyhead{}
  \fancyfoot{}
  \renewcommand{\headrulewidth}{0.0pt}
}


% Remove headers from empty pages
%-----------------------------------------------------------------------------------
\makeatletter
\def\cleardoublepage{\clearpage\if@twoside \ifodd\c@page\else%
    \hbox{}%
    \thispagestyle{empty}%              % Empty header styles
    \newpage%
    \if@twocolumn\hbox{}\newpage\fi\fi\fi}
\makeatother



% Remove number from first page of a chapter
%------------------------------------------------------------------------------------
\usepackage{etoolbox}
\patchcmd{\chapter}{plain}{empty}{}{}
\makeatletter
  \let\ps@plain\ps@empty
\makeatother

% Add nice chapter quote environment
%------------------------------------------------------------------------------------
\makeatletter
%\renewcommand{\@chapapp}{}% Not necessary...
\newenvironment{chapquote}[2][2em]
{\setlength{\@tempdima}{#1}%
	\def\chapquote@author{#2}%
	\parshape 1 \@tempdima \dimexpr\textwidth-2\@tempdima\relax%
	\itshape}
{\par\normalfont\hfill--\ \chapquote@author\hspace*{\@tempdima}\par\bigskip}
\makeatother

%hyperref, breqn e altri pacchetti rompipalle
%-----------------------------------------------------------------------------------
\usepackage[linktocpage,bookmarksnumbered,pagebackref]{hyperref} % clickable links, citations and crossreferences. Can change their colour and appearance in the options.
\usepackage{breqn} % equations on more lines, remove if no equations
\usepackage{pdfpages} 

\usepackage[acronym, nonumberlist]{glossaries}
\renewcommand{\glsnamefont}[1]{\textbf{#1}}
