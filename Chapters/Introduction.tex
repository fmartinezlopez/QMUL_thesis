\chapter{Introduction}
\label{chapter:introduction}

%\begin{chapquote}{Stephen King, \textit{On Writing: A Memoir of the Craft}}
%	The scariest moment is always just before you start. After that, things can only get better.
%\end{chapquote}

\begin{chapquote}{Plato, \textit{The Republic}}
	The beginning is the most important part of any work.
\end{chapquote}

% The SM and what it can't explain
The Standard Model (\gls{sm}) of particle physics \cite{Glashow1961,Weinberg1967,Salam1968} has provided a deep understanding of the electromagnetic, weak and strong interactions, and over the past decades it has passed all kind of precision tests \cite{Erler2019}. However, the \gls{sm} by itself cannot explain certain observed phenomena, such as the baryon asymmetry of the universe \cite{Canetti2012}, the existence of dark matter (\gls{dm}) \cite{Bertone2004}, or the origin of neutrino masses \cite{King2014}.

% CP violation
One of the biggest puzzles in physics nowadays is how the universe came to be matter-dominated. Following the Big Bang, matter and antimatter were created in equal amounts. A. D. Sakharov described what are the necessary conditions to generate a matter-antimatter asymmetry in the early universe \cite{Sakharov1967}. One of them is the existence of interactions that violate the charge-parity (\gls{cp}) symmetry. It has already been established that the amount of \gls{cp} violation in the quark sector is not enough to generate the baryon asymmetry \cite{Gavela1993}. Leptons could contribute to the \gls{cp} violation through the neutrino oscillation mechanism \cite{Akhmedov1998}. However, there is no experimental evidence for this so far.

% Dark Matter
Another yet to be solved mystery of modern physics concerns the nature of \gls{dm}. From astrophysical observations (see Ref. \cite{Bertone2004} and references therein), we are aware of the existence of some unknown matter which only interacts gravitationally with other particles. Usually, extensions of the \gls{sm} include feasible \gls{dm} candidates. These are usually very stable, heavy particles with small interactions (if any) with \gls{sm} particles. These states are known as weakly interacting massive particles (\gls{wimp}s) \cite{Lee1977,Jungman1995}. Experiments looking for \gls{dm} have constrained the interaction cross section between \gls{dm} and \gls{sm} particles to be very small for \gls{dm} masses below 1 TeV \cite{Arcadi2024}.

% DUNE
Among other next generation particle physics experiments, the Deep Underground Neutrino Experiment (\gls{dune}) stands out. Conceived as a neutrino oscillation experiment, it will provide definitive answers to different open questions in the neutrino sector. Its main goals are the discovery of \gls{cp} violation in the leptonic sector and the determination of the neutrino mass ordering \cite{DUNE2020TDR1}. It will also provide precision measurements of the oscillation parameters within the three-flavour picture.

% More on DUNE (?)
The \gls{dune} far detector (\gls{fd}) will also search for baryon-number violation and neutrinos originated from supernova explosions. Moreover, its near detector (\gls{nd}) complex will sit next to the most powerful neutrino beam to date, allowing for a rich neutrino cross section programme. This broad physics range requires a superb performance from the detectors, which can also be used to look for other \gls{bsm} phenomena.

\begin{comment}
% One paragraph summary of each topic covered
% DUNE FD DAQ - Matched filter
In this thesis, I explore three different aspects of \gls{dune}. Focusing on the data acquisition system of the far detector, I start by proposing a method to enhance the sensitivity of the online processing to low energy events. The idea is to modify the processing chain in order to have more information available to form trigger decisions. I motivate this new approach using both ProtoDUNE data and Monte Carlo (MC) samples, as well as with the results from a test in a real detector setup.

% DUNE FD BSM - Solar DM
Then, I investigate the potential of detecting neutrino fluxes from \gls{dm} annihilations inside the Sun with \gls{dune}. Although this is the territory of the large volume neutrino telescopes, a detector with the high resolution and pointing capabilities of the \gls{dune} \gls{fd} can provide complementary information in certain regimes. I present here the results of a preliminary analysis, showing the projected sensitivities for the general case and two particular \gls{dm} scenarios.

% DUNE ND sim/reco - ND-GAr particle ID
Finally, I discuss my work on the reconstruction of \gls{ndgar}, the gaseous argon component of the \gls{dune} \gls{nd}. These efforts were focused towards the development of the particle identification strategy in the detector. Following a series of additions and upgrades in the reconstruction, I make use of that to perform the first event selection studies with fully reconstructed events in this detector.
\end{comment}

In this thesis, I explore three different aspects of \gls{dune}. First, I propose a method to enhance the sensitivity of the online processing in the \gls{fd} to low energy events. Then, I investigate the potential of detecting neutrino fluxes from \gls{dm} annihilations inside the Sun with the \gls{dune} \gls{fd}. Next, I discuss my work on the reconstruction of \gls{ndgar}, the gaseous argon component of the \gls{dune} \gls{nd}. Finally, I make use of those upgrades to perform the first event selection studies with fully reconstructed events in this detector.

\begin{comment}
% Thesis overview
% Neutrino physics
This thesis opens with an overview of the status of neutrino physics in Chapter \ref{chapter:neutrinos}. I start summarising the role that neutrinos play in the \gls{sm}, to then focus on the developments that lead to the discovery of neutrino oscillations and how to accommodate massive neutrinos in the model. I then discuss the phenomenology of the neutrino oscillations, as well as the current experimental landscape and open questions. In the final section, I review the basics of the neutrino-nucleus interaction modelling, which is of great importance for \gls{dune}.

% DUNE
Chapter \ref{chapter:dune} introduces \gls{dune}, its physics programme and various components. I give detail descriptions of the LBNF beamline, the near detector and the far detector designs. I also discuss the current staging plans for \gls{dune}. This leads to the of \gls{ndgar}, the more capable near detector planned for \gls{dune} Phase II.

% Matched filter
In Chapter \ref{chapter:matched_filter} I start by reviewing how the trigger primitives (\gls{tp}s), the basic building blocks of the \gls{dune} far detector trigger chain, are formed. I then motivate how to use the filtering to enhance the TP generation in the induction channels. I describe the concept of matched filter, and how to optimise it using ProtoDUNE-SP data. I use different MC samples to study its performance, and assess how it improves the hit finding. Finally, I present the results of the tests we performed at the VD ColdBox setup at CERN, were for the first time we collected TP data with a matched filter.

%Solar DM
The solar \gls{dm} analysis is presented in Chapter \ref{chapter:dm_analysis}. After reviewing the theoretical basis for the solar \gls{dm} capture and how capture and annihilation rates are related, I introduce the analysis framework used. I then focus on the event selection studies based on two topologies: high-energy \gls{dis} events and low-energy $1\mu1p$ QE events. I use these to extract the projected sensitivities for the \gls{dm}-nucleon scattering cross section, and compare them to the current status of other direct and indirect \gls{dm} searches. Additionally, I discuss the potential of \gls{dune} in two specific \gls{dm} models. I end with a discussion of the systematic uncertainties relevant for this analysis.

% ND-GAr PID
Chapter \ref{chapter:garsoft_pid} starts with a description of GArSoft, the simulation and reconstruction software of \gls{ndgar}. Then, I describe the charge calibration procedure I implemented using a MC sample of stopping protons. I use this to compute the mean ionisation loss per unit length of the tracks, and show how this procedure can be used for particle separation. Next, I summarise my investigations on the muon and pion separation using the information from the calorimeter. I outline the strategy I followed for the training and testing of the classifiers, commenting on the achieved performance using a neutrino interaction sample. Following this, I introduce the possibility of using the fast timing of the calorimeter to perform a time-of-flight measurement. It will allow to separate pions and protons in a momentum range not accessible to other methods. Additionally, I present a method to identify the decays of charged pions in the TPC, when the decay angle is too small and the pion and muon get merged into a single track. I construct a collection of variables from the track fit that allow to locate the position of the decay. Lastly, I propose a new clustering algorithm optimised for our calorimeter. I then demonstrate its impact in the context of the neutral pion reconstruction. The Chapter finishes with an overview of the integration of these reconstruction items in GArSoft.

% ND-GAr selection
The event selection studies are covered in Chapter \ref{chapter:gar_selection}. I start by describing the MC neutrino interaction sample I use for the studies. Then, I focus on the $\nu_{\mu}$ \gls{cc} selection, which includes an optimisation of the fiducial volume. I also explore the kinematics of the selected primary muon and the location of the neutrino vertex. Next, I study the performance of the selections based on the reconstructed charged pion multiplicity, paying special attention to the $1\pi^{\pm}$ selection. I briefly discuss the possibility of adding the neutral pions in the analysis. Following that, I present the results on the energy reconstruction for the selected charged-current events. I finish with a detailed discussion of the different sources of systematic error relevant for \gls{ndgar}. These include flux and neutrino interaction modelling uncertainties, as well as detector effects.

% Conclusions
Eventually, the thesis concludes with Chapter \ref{chapter:conclusion}. There, I summarise the main results presented in this work, and discuss future plans for the different projects.
\end{comment}

This thesis opens with an overview of the status of neutrino physics in Chapter \ref{chapter:neutrinos}. Chapter \ref{chapter:dune} introduces \gls{dune}, its physics programme and various components, including \gls{ndgar}. In Chapter \ref{chapter:matched_filter} I review the possibility of using matched filters to form online hits in the \gls{fd}. The solar \gls{dm} analysis is presented in Chapter \ref{chapter:dm_analysis}. Chapter \ref{chapter:garsoft_pid} describes the work on the \gls{ndgar} reconstruction, focused on particle identification. The event selection studies in \gls{ndgar} are covered in Chapter \ref{chapter:gar_selection}. Eventually, the thesis concludes with Chapter \ref{chapter:conclusion}, where I discuss future plans for the different projects.