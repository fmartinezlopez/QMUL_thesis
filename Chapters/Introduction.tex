\chapter{Introduction}
\label{chapter:introduction}

% The SM and what it can't explain
The Standard Model of particle physics (SM) has provided a deep understanding of the electromagnetic, weak and strong interactions, and over the past decades it has passed all kind of precision tests. However, the SM by itself can not explain certain observed phenomena, such as the baryon asymmetry of the Universe, the nature of Dark Matter (DM), or the origin of neutrino masses.

% DUNE
Among other next generation particle experiments, the Deep Underground Neutrino Experiment (DUNE) stands out. Conceived as a neutrino oscillation experiment, it will provide definitive answers to different open questions in the neutrino sector. Its main goals are the discovery of CP violation in the leptonic sector and the determination of the neutrino mass ordering \cite{DUNE2020TDR1}. It will also provide precision measurements of the oscillation parameters within the three-flavour picture.

The DUNE detectors will also search for baryon-number violation and neutrinos originated from supernova explosions (SNB). This broad physics scope requires a superb performance of the detectors, which can be used to look for other BSM phenomena. Its near detector complex , allowing for a rich neutrino cross section programme.

% One paragraph summary of each topic covered
% DUNE FD DAQ - Matched filter
In this thesis, I explore three different aspects of DUNE. Focusing on the data acquisition system of the far detector, I start by proposing a method to enhance the sensitivity of the online processing to low energy events. The idea is to modify the processing chain in order to have more information available to form trigger decisions. I motive this new approach using both ProtoDUNE data and Monte Carlo samples, and had the opportunity to test it in a real setup.

% DUNE FD BSM - Solar DM
Then, I investigate the potential of detecting neutrino fluxes from DM annihilations inside the Sun with DUNE. Although this is the territory of the large volume neutrino telescopes, a detector with the high resolution and pointing capabilities of the DUNE FD can provide complementary information in certain regimes. I present here the results of a preliminary analysis, showing the projected sensitivities for the general case and two particular DM scenarios.

% DUNE ND sim/reco - ND-GAr particle ID
Finally, I discuss my work on the reconstruction of ND-GAr, the gaseous argon component of the DUNE ND. my efforts towards a particle identification strategy in the detector first event selection studies using fully reconstructed events

% Thesis overview
% Neutrino physics
This thesis opens with an overview of the status of neutrino physics in Chapter \ref{chapter:neutrinos}. I start summarising the role that neutrinos play in the SM, to then focus on the developments that lead to the discovery of neutrino oscillations and how to accommodate massive neutrinos in the model. I then discuss the phenomenology of the neutrino oscillations, as well as the current experimental landscape and open questions. In the final section, I review the basics of the neutrino-nucleus interaction modelling, which is of great importance for DUNE.

% DUNE
Chapter \ref{chapter:dune} introduces DUNE, its physics programme and various components. I give detail descriptions of the LBNF beamline, the near detector and the far detector designs. I also the current staging plans for DUNE, which 

% Matched filter
In Chapter \ref{chapter:matched_filter} I start by reviewing how the trigger primitives, the basic building blocks of the DUNE far detector trigger chain, are formed. present the studies I performed 

% ND-GAr PID
Chapter \ref{chapter:garsoft_pid}

% ND-GAr selection
Chapter \ref{chapter:gar_selection}

% Conclusions
Eventually, the thesis concludes with Chapter \ref{chapter:conclusion}. There, I summarise the main results presented in this work, and discuss future plans for the different projects.