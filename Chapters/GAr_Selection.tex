\chapter{Event selection in ND-GAr}
\label{chapter:gar_selection}

\section{Data sample}

{\color{red}
In this section I need to make sure to mention:
\begin{itemize}
    \item I need to comment on the versions of the software that were used for the production of the different samples (if we end up having more than one). The version of \texttt{GENIE} used was 
    \item We use GArG4 instead of \texttt{edep-sim} for the particle propagation. Because both \texttt{Geant4} wrappers use different configurations for the simulation, the results obtained are different. The default \texttt{edep-sim} configuration used by the DUNE ND is appropriate for ND-LAr, where thresholds for particle production are higher. In the case of ND-GAr, these parameters need to be adjusted accordingly. For the time being, in these first productions of analysis files, we will use our standalone \texttt{Geant4} implementation. For future iterations these differences will need to be revisited and understood, so we can use the same simulation workflow as the rest of the ND.
    \item I need to comment on the sample size. The first sample produced was simply $10^{5}$ events inside the HPgTPC volume. There is also the question of the other sample we may want to produce for the $\geq 3\pi^{\pm}$ selection (ask Naseem).
    \item So far we have only simulated single interaction events. Ideally, we should move to simulate full spills. Of course, we need to understand how many interactions we expect in ND-GAr per spill. Also, there is the question of having neutrino interactions happening in the other detector volumes (ECal, magnet, \dots).
    \item At some point, we should generate a sample of rock muons making it to ND-GAr.
    \item I think I should comment on the run plan (at least the part that concerns ND-GAr), and what it means in terms of generating a full production sample. It will be good to have an understanding of the POT we need on-axis and at each off-axis positions (for both FHR and RHC).
\end{itemize}
}

\begin{table}[t]
	\caption{Event rates in ND-GAr.}
	\begin{center}
		\begin{small}
			\begin{tabular}{lcc}
                & \multicolumn{2}{c}{Events/ton/year}                                                                   \\[2mm] \cline{2-3} 
            \multicolumn{1}{c}{\rule{0pt}{1.1\normalbaselineskip}Process}       & $1.1 \times 10^{21} ~ \mathrm{POT}/\mathrm{year}$ & $1.9 \times 10^{21} ~ \mathrm{POT}/\mathrm{year}$ \\[2mm] \hline
            \rule{0pt}{1.1\normalbaselineskip}All $\nu_{\mu}$-CC                & $1.60 \times 10^{6}$                              & $2.83 \times 10^{6}$                              \\[2mm]
            $\hspace{0.5 cm}$ CC $0\pi$       & $5.28 \times 10^{5}$                              & $9.35 \times 10^{5}$                              \\[2mm]
            $\hspace{0.5 cm}$ CC $1\pi^{\pm}$ & $3.02 \times 10^{5}$                              & $5.34 \times 10^{5}$                              \\[2mm]
            $\hspace{0.5 cm}$ CC $1\pi^{0}$   & $1.65 \times 10^{5}$                              & $2.92 \times 10^{5}$                              \\[2mm]
            $\hspace{0.5 cm}$ CC $2\pi$       & $3.18 \times 10^{5}$                              & $5.63 \times 10^{5}$                              \\[2mm]
            $\hspace{0.5 cm}$ CC $3\pi$       & $1.36 \times 10^{5}$                              & $2.41 \times 10^{5}$                              \\[2mm]
            $\hspace{0.5 cm}$ CC other        & $1.52 \times 10^{5}$                              & $2.69 \times 10^{5}$                              \\[2mm] \hline
            \rule{0pt}{1.1\normalbaselineskip}All $\bar{\nu}_{\mu}$-CC          & $7.54 \times 10^{4}$                              & $1.33 \times 10^{5}$                              \\[2mm]
            All NC                            & $5.50 \times 10^{5}$                              & $9.73 \times 10^{5}$                              \\[2mm]
            All $\nu_{e}$-CC                  & $2.70 \times 10^{4}$                              & $4.78 \times 10^{4}$                             
            \end{tabular}
		\end{small}
	\end{center}
	\label{tab:ndgar_event_rates}
\end{table}

\section[\texorpdfstring{$\nu_{\mu}$}{numu} CC selection]{\boldmath\texorpdfstring{$\nu_{\mu}$}{numu} CC selection}

In a $\nu_{\mu}$ CC inclusive selection, the signal topology we look for is a neutrino-induced muon with or without other final state particles. Here, I also require the neutrino vertex to be located inside the fiducial volume (FV) of ND-GAr.

The FV is defined as a smaller cylinder within the cylindrical volume of the HPgTPC. The FV has a radius $R_{\mathrm{FV}}$ and a half-length $L_{\mathrm{FV}}$. For a particle position to lie within the FV it must satisfy:
\begin{equation}
    \vec{x}_{i} \in \left\{\vec{x} \in \mathbb{R}^{3} \mid |x_{0}| \leq L_{\mathrm{FV}} ~ \& ~ \sqrt{x_{1}^{2}+x_{2}^{2}} \leq R_{\mathrm{FV}}\right\},
\end{equation}
in the reference frame of the HPgTPC. For convenience, I define:
\begin{equation}
    \begin{split}
        \Delta R_{\mathrm{FV}} &= R_{\mathrm{HPgTPC}} - R_{\mathrm{FV}}, \\
        \Delta L_{\mathrm{FV}} &= L_{\mathrm{HPgTPC}} - L_{\mathrm{FV}},
    \end{split}
\end{equation}
where $R_{\mathrm{HPgTPC}}$ and $L_{\mathrm{HPgTPC}}$ refer to the radius and the half-length of the HPgTPC, respectively. Figure \ref{fig:ndgar_ana_geometry} shows the HPgTPC volume with the FV inside of it. In that representation, the FV is defined as $\Delta L_{\mathrm{FV}} = 30.0 ~ \mathrm{cm}$ and $\Delta R_{\mathrm{FV}} = 30.0 ~ \mathrm{cm}$. Also shown is the HPgTPC reference frame, with $x$ being the drift direction and $z$ aligned along the beam direction.

In some cases, it is interesting to divide the signal events in different categories based on their true interaction mode. In this work, I will distinguish between charged-current quasi-elastic (CCQE), coherent (CCCOH), resonant (CCRES), and deep-inelastic (CCDIS) interactions. I also use a separate category for the interactions not included in any of the other categories (CCOther).

Any other events are considered backgrounds. For this selection, I use the following categorisation of background events:
\begin{itemize}
    \item Out of FV: if the true neutrino vertex lies outside the defined FV.
    \item NC: if the event is a true neutral-current event.
    \item $\bar{\nu}_{\mu}$ CC: if the true neutrino candidate is of muon antineutrino flavour.
    \item Other: if the event is not signal nor falls in any of the other background categories.
\end{itemize}

\begin{figure}[t]
\centering
\includegraphics[width=.90\linewidth]{Images/GAr_selection/ndgar_ana_geometry.pdf}
\caption[Schematic diagram of the HPgTPC including the fiducial volume.]{Schematic diagram of the HPgTPC including the fiducial volume (FV). In this case the FV is given by $\Delta L_{\mathrm{FV}} = 30.0 ~ \mathrm{cm}$ and $\Delta R_{\mathrm{FV}} = 30.0 ~ \mathrm{cm}$.}
\label{fig:ndgar_ana_geometry}
\end{figure}

The key to the CC selection is the identification of a primary muon candidate. Typically, this is the longest track in the event. However, sometimes protons and pions leave tracks longer than that of the muon. This is particularly important in the GAr medium, considerably less dense than the LAr. For this reason, the muon identification in ND-GAr relies heavily on the capabilities of the ECal.

The selection strategy proposed combines the information coming from the three main detection systems of ND-GAr: the HPgTPC charge readout, and the ECal and $\mu$ID detectors. It consists of five steps:
\begin{enumerate}
    \item Event contains reconstructed particles.
    \item Select particles with reconstructed negative charge, $q_{\mathrm{reco}} = -1$.
    \item Select particles passing the muon score cut, $\mu_{\mathrm{score}} \geq \mu_{\mathrm{score}}^{\mathrm{cut}}$.
    \item Keep reconstructed particle with the highest momentum, $\mathrm{max}\left[p_{\mathrm{reco}}\right]$.
    \item Check that the remaining particle starts within the FV.
\end{enumerate}
All the events passing these cuts are classified as signal, and the selected particle is regarded as the primary muon candidate.

\begin{figure}[t]
    \centering
    \includegraphics[width=.99\linewidth]{Images/GAr_selection/true_numu_example_spectra_horizontal.pdf}
    \caption[True positive, false positive, and false negative true neutrino energy distributions for a $\nu_{\mu}$ CC selection.]{True positive (left panel), false positive (middle panel), and false negative (right panel) true neutrino energy distributions for the $\nu_{\mu}$ CC selection given by a muon score cut of $\mu_{\mathrm{score}}^{\mathrm{cut}} = 0.75$, and a FV defined as $\Delta L_{\mathrm{FV}} = 30.0 ~ \mathrm{cm}$ and $\Delta R_{\mathrm{FV}} = 30.0 ~ \mathrm{cm}$.}
    \label{fig:numuCC_spectra_example}
\end{figure}

\subsection{Selection optimisation}

I performed an optimisation of this selection, comparing the performance of a number of configurations. For the muon selection, I varied the value of $\mu_{\mathrm{score}}^{\mathrm{cut}}$ from $0.05$ to $0.95$, using a step size of $0.05$. Additionally, to optimise the FV, I systematically explored a number of different parameter configurations, moving within the $10.0-70.0~\mathrm{cm}$ range for $\Delta L_{\mathrm{FV}}$ and $25.0-75.0~\mathrm{cm}$ for $\Delta R_{\mathrm{FV}}$, in increments of $10.0~\mathrm{cm}$ and $5.0~\mathrm{cm}$ respectively.

For each parameter configuration, I extract three different true neutrino energy distributions. These are built combining the results of the selection described previously, which we can refer to as the ``reco'' selection, and a ``true'' selection. The later identifies the true $\nu_{\mu}$ CC events using the GENIE event records, and checks that the true neutrino vertices are contained in the FV.

The first distribution consists of the events passing both selections, i.e., these are the true $\nu_{\mu}$ CC events which pass the ``reco'' selection. The second distribution contains the events passing the ``reco'' selection but failing the ``true'' selection. These are the background events that the selection misidentifies. Finally, the third distribution corresponds to the events picked by the ``true'' selection but not by the ``reco'' one. In other words, these are the true $\nu_{\mu}$ CC events that our selection misses. In analogy to the machine learning jargon, I refer to these distributions as the true positive (TP), false positive (FP), and false negative (FN) spectra, respectively. Figure \ref{fig:numuCC_spectra_example} shows an example of these three distributions for the case $\mu_{\mathrm{score}}^{\mathrm{cut}} = 0.75$, $\Delta L_{\mathrm{FV}} = 30.0 ~ \mathrm{cm}$, and $\Delta R_{\mathrm{FV}} = 30.0 ~ \mathrm{cm}$.

By making different combinations of these distributions one can compute a series of performance metrics. Using the full information from the spectra allows to obtain the scores as a function of the true neutrino energy, whereas the totals can be obtained by integrating the histograms. This way, the efficiency of the selection is given by:
\begin{equation}
    \begin{split}
        \mathrm{Efficiency} &= \frac{\text{Selected true } \nu_{\mu} \text{ CC events}}{\text{Total true } \nu_{\mu} \text{ CC events}}\\
        &= \frac{\mathrm{TP}}{\mathrm{TP}+\mathrm{FN}},
    \end{split}
\end{equation}
while the purity can be written as:
\begin{equation}
    \begin{split}
        \mathrm{Purity} &= \frac{\text{Selected true } \nu_{\mu} \text{ CC events}}{\text{Total selected events}}\\
        &= \frac{\mathrm{TP}}{\mathrm{TP}+\mathrm{FP}}.
    \end{split}
\end{equation}

Another scoring metric typically used when quantifying the performance of a selection is the significance. It is defined as:
\begin{equation}
    \mathrm{Significance} = \frac{S}{\sqrt{S+B}} = \frac{\mathrm{TP}}{\sqrt{\mathrm{TP} + \mathrm{FP}}}.
\end{equation}
The significance measures the relative size of the true signal within the selection, $S=\mathrm{TP}$ with respect to one standard deviation of the counting experiment. Assuming Poisson statistics, the variance is equal to the number of observations, and therefore the standard deviation equals to $\sqrt{N}=\sqrt{S+B}=\sqrt{\mathrm{TP} + \mathrm{FP}}$. I use this metric to 

\begin{figure}[t]
    \centering
    \includegraphics[width=.99\linewidth]{Images/GAr_selection/efficiency_and_purity_error_boxes.pdf}
    \caption[Efficiency and purity for the $\nu_{\mu}$ CC selection as a function of the muon score cut, FV half-length cut, and radial cut.]{Efficiency (blue) and purity (red) for the $\nu_{\mu}$ CC selection as a function of the muon score cut (left panel), FV half-length cut (middle panel), and radial cut (right panel). The height of the boxes represents the IQR of the conditional distributions, whereas the horizontal line corresponds to the mean.}
    \label{fig:numuCC_metrics_opt}
\end{figure}

\begin{figure}[t]
    \centering
    \includegraphics[width=.99\linewidth]{Images/GAr_selection/significance_error_boxes.pdf}
    \caption[Significance for the $\nu_{\mu}$ CC selection as a function of the muon score cut, FV half-length cut, and radial cut.]{Significance for the $\nu_{\mu}$ CC selection as a function of the muon score cut (left panel), FV half-length cut (middle panel), and radial cut (right panel). The height of the boxes represents the IQR of the conditional distributions, whereas the horizontal line corresponds to the mean.}
    \label{fig:numuCC_significance_opt}
\end{figure}

Figure \ref{fig:numuCC_metrics_opt} shows the change in efficiency (blue) and purity (red) of the $\nu_{\mu}$ CC selection as a function of the different cuts. From left to right, I vary $\mu_{\mathrm{score}}^{\mathrm{cut}}$, $\Delta L_{\mathrm{FV}}$, and $\Delta R_{\mathrm{FV}}$. For each value of the cuts, I compute the median and IQR (represented by the horizontal lines and the heights of the boxes, respectively) of the corresponding conditional distributions of efficiency and purity. This representation is useful to get an idea of the general trend the scores follow with the cuts, as well as the spread. It is clear that the muon score cut has the biggest impact on the efficiency, which ranges between $0.05$ to $0.80$, whereas the purity remains stable with values around $0.85$.

\begin{figure}[t]
    \centering
    \includegraphics[width=.95\linewidth]{Images/GAr_selection/numuCC_metric_heatmap.pdf}
    \caption{Normalised 2D distributions of efficiency, purity and significance for the $\nu_{\mu}$ CC selection. The $S/\sqrt{S+B}$ is normalised to the highest value achieved. The vertical dashed line indicates a purity value of $0.85$, whereas the horizontal one corresponds to an efficiency of $0.80$.}
    \label{fig:numuCC_metric_heat}
\end{figure}

A similar depiction of the significance can be found in Fig. \ref{fig:numuCC_significance_opt}. In this case, one can see that the $S/\sqrt{S+B}$ decreases as the cuts grow tighter. However, there are hints of local maxima at intermediate values.

Selecting the cut configuration with the highest significance, $147 \pm 11$ for the parameter values explored here, results in an efficiency and purity of $0.754 \pm 0.006$ and $0.833 \pm 0.007$, respectively. Figure \ref{fig:numuCC_metric_heat} shows the 2D distributions resulting when combining pairs of efficiency, purity and significance, obtained for the cut configurations explored. The significance is normalised to the highest value obtained in the parameter scan. Looking at this, it is clear that a selection with highest efficiency and purity can be achieved, maintaining a similar significance level.

\begin{figure}[t]
	\centering
	\includegraphics[width=.90\linewidth]{Images/GAr_selection/numu_cc_selection_steps.pdf}
	\caption[Cumulative efficiency and purity for the $\nu_{\mu}$ CC selection.]{Cumulative efficiency (blue) and purity (red) of the $\nu_{\mu}$ CC selection. The secondary axis indicates the number of events in the sample after each cut (black crosses).}
	\label{fig:numuCC_selection_steps}
\end{figure}

\begin{table}[h!]
	\caption[Step-by-step $\nu_{\mu}$ CC selection cuts and cumulative passing rates.]{Step-by-step $\nu_{\mu}$ CC selection cuts and cumulative passing rates. Relative passing rates are indicated in parentheses.}
	\begin{center}
		\begin{small}
			\begin{tabular}{c|ccc}
                Cut \# & Selection cut                       & Events & Passing rates          \\[2mm] \hline
                \rule{0pt}{1.1\normalbaselineskip}0      & Total number of events (No cuts)    & 100000 & $100.00\% ~(100.00\%)$ \\[2mm]
                1      & At least one reconstructed particle                              & 85680  & $85.68 \% ~(85.68 \%)$ \\[2mm]
                2      & Negatively charged particles only                                & 62054  & $62.05\% ~(72.43\%)$   \\[2mm]
                3      & $\mu_{\mathrm{score}} \geq \mu_{\mathrm{score}}^{\mathrm{cut}}$  & 46585  & $46.59\% ~(75.07\%)$   \\[2mm]
                4      & Candidate $\vec{x}_{\mathrm{start}}$ in FV                       & 31834  & $31.83\% ~(68.34\%)$  
                \end{tabular}
		\end{small}
	\end{center}
	\label{tab:numuCC_selection}
\end{table}

Therefore, to get a more refined selection, I first select the configurations with a purity and an efficiency higher than $0.85$ and $0.80$, respectively. After that, I select the tuple of cuts yielding the highest significance. The resulting value for the muon score cut is $\mu_{\mathrm{score}}^{\mathrm{cut}} = 0.10$, and the FV is given by $\Delta L_{\mathrm{FV}} = 30.0~\mathrm{cm}$ and $\Delta R_{\mathrm{FV}} = 50.0~\mathrm{cm}$. With these, one obtains a total efficiency of $0.800 \pm 0.007$ and purity of $0.851 \pm 0.008$, with a significance of $138 \pm 11$. Hereafter, I use this optimised selection cuts, unless specified otherwise.

A summary of the selection can be found in Tab. \ref{tab:numuCC_selection}. It shows the number of events in the selected sample after each selection cut, as well as the absolute and relative passing rates. Figure \ref{fig:numuCC_selection_steps} shows the overall efficiencies (blue) and purities (red) after each cut in the event selection is applied. As expected, the efficiency drops while the purity increases with the successive cuts.

Notice how, out of the cuts prior to the FV constraint, the sign selection produces the highest increase in purity. This is one of the advantages of having a magnetised TPC, and can also be used for a $\bar{\nu}_{\mu}$ CC selection when running in RHC mode.

\begin{figure}[t]
	\centering
	\includegraphics[width=.80\linewidth]{Images/GAr_selection/numuCC_selection_true_energy.pdf}
	\caption[True neutrino energy spectra for the $\nu_{\mu}$ CC selection.]{True neutrino energy spectra for the $\nu_{\mu}$ CC selection. The selected events correspond to the coloured stacked histogram, broken down by signal and background subcategories. The statistical uncertainty is drawn in hatched gray. The true distribution is also shown with the black data points. The bottom panel shows the ratio between the number of true and selected $\nu_{\mu}$ CC events per bin.}
	\label{fig:numuCC_selection_true_enu}
\end{figure}

\begin{figure}[t]
	\centering
	\includegraphics[width=.99\linewidth]{Images/GAr_selection/numuCC_selection_true_energy_performance.pdf}
	\caption[True neutrino energy spectra for the $\nu_{\mu}$ CC selection.]{Left panel: efficiency (top panel) and purity (bottom panel) for the $\nu_{\mu}$ CC selection as a function of the true neutrino energy. Right panel: significance for the $\nu_{\mu}$ CC selection as a function of the true neutrino energy}
	\label{fig:numuCC_selection_true_enu_performance}
\end{figure}

\subsection{Selection performance}

Using the stored spectra discussed above, the true neutrino energy distribution for the selected events can be recovered doing $\mathrm{TP}+\mathrm{FP}$. Similarly, the combination $\mathrm{TP}+\mathrm{FN}$ gives the true spectrum. Figure \ref{fig:numuCC_selection_true_enu} shows the true (black data points) and selected (coloured stacked histogram) $E_{\nu}$ distributions for the optimised $\nu_{\mu}$ CC selection. The colours in the selected spectrum indicate the different signal categories and backgrounds, with the overall statistical uncertainty represented by the gray hatched mess. The ratio between the true and selected events is also shown. One can see that it sits around $1$ for most of the energy range. However, for energies $\leq 1~\mathrm{GeV}$ there is a significant deficit of selected events.

\begin{figure}[t]
	\centering
	\includegraphics[width=.80\linewidth]{Images/GAr_selection/numuCC_selection_true_energy_sign_comp.pdf}
	\caption{True neutrino energy spectra for the $\nu_{\mu}$ CC selection with (blue) and without (red) sign selection. The selected events are broken down by true positives (signal) and false positives (background). The true distribution is also shown (black data points). The bottom panel shows the ratios between the number of false positives and total selected events per bin.}
	\label{fig:numuCC_sign_selection}
\end{figure}

These spectra also allow to compute the efficiency and purity of the selection as a function of the true neutrino energy, as shown in Fig. \ref{fig:numuCC_selection_true_enu_performance} (left panel). As it could be expected from the previous ratio plot, the efficiency is low at low neutrino energies. Nonetheless, it raises quickly with the energy, until it stabilises around a value of $0.80$. Looking at the purity, one may notice that, although it starts at around $0.90$, there is a significant decrease towards the high end of the spectrum. Figure \ref{fig:numuCC_selection_true_enu_performance} (right panel) also shows the significance as a function of the energy. In this case, the highest $S/\sqrt{S+B}$ is achieved around the energies where the spectrum peaks.

A variation of the $\nu_{\mu}$ CC selection one can try is to apply it without the reconstructed charge cut. Figure \ref{fig:numuCC_sign_selection} (top panel) shows the $E_{\nu}$ distributions corresponding to the selection with (blue stacked histogram) and without (red stacked histogram) the sign selection. In the former case, the out of FV contamination amounts to $9.06\%$ of the total, while the NC contamination results $4.77\%$ and the wrong-sign contamination $0.57\%$. For the later, these backgrounds account for the $10.01\%$, $10.82\%$, and $2.18\%$ of the selected events, respectively. As expected, removing the positive particles does not change the FV-related effects noticeably. However, the sign selection proves its worth in the rejection of $\bar{\nu}_{\mu}$ CC events, which drop almost by one order of magnitude. Additionally, the charge selection cuts the NC events in half, as it reduces the chances of misidentifying a positively charged hadron for a muon.

\begin{comment}
\begin{figure}[t]
	\begin{subfigure}{0.5\textwidth}
		\centering
		\includegraphics[width=.99\linewidth]{Images/GAr_selection/nu_energy_with_breakdown.pdf}
	\end{subfigure}
	\begin{subfigure}{0.5\textwidth}
		\centering
		\includegraphics[width=.99\linewidth]{Images/GAr_selection/nu_energy_with_breakdown_no_sign.pdf}
	\end{subfigure}
	\caption[True neutrino energy spectra for the $\nu_{\mu}$ CC selection with and without sign selection.]{True neutrino energy spectra for the $\nu_{\mu}$ CC selection with (left panel) and without (right panel) sign selection. The selected events are broken down by signal or background category. The true distribution is also shown (black data points).}
	\label{fig:numuCC_sign_selection}
\end{figure}
\end{comment}

\begin{figure}[t]
    \centering
    \includegraphics[width=.99\linewidth]{Images/GAr_selection/numuCC_muon_kinematic_comp.pdf}
    \caption{Distributions for the reconstructed versus truth generated primary muon momentum (top left panel), longitudinal momentum (top right panel), transverse momentum (bottom left panel), and beam angle (bottom right panel). The reconstructed values correspond to the selected primary muon candidate, whereas the truth values come from the true primary muon in the event.}
    \label{fig:numuCC_muon_kinematic_comp}
\end{figure}

\subsection{Primary muon kinematics}

Figure \ref{fig:numuCC_muon_kinematic_comp} shows a comparison between some of the reconstructed and truth primary muon kinematic variables. Notice that, for the reconstructed values, the . That means that, in some cases, we are comparing the kinematic, i.e. the account for both reconstruction and selection deficiencies.

study the performance of the $\nu_{\mu}$ CC selection as a function of the kinematic variables of the primary muon.

\begin{figure}[t]
	\centering
	\includegraphics[width=.99\linewidth]{Images/GAr_selection/numuCC_selection_true_kinematics_performance.pdf}
	\caption[Efficiency and purity of the $\nu_{\mu}$ CC selection as a function of the primary muon true momentum and beam angle.]{Efficiency (blue) and purity (red) of the $\nu_{\mu}$ CC selection as a function of the primary muon true momentum (left panel) and beam angle (right panel).}
	\label{fig:numuCC_muon_kinematics}
\end{figure}

\begin{figure}[t]
    \centering
    \includegraphics[width=.99\linewidth]{Images/GAr_selection/numuCC_true_vertex_position_fiducial.pdf}
    \caption[Distributions of the true $\nu_{\mu}$ CC vertex positions for the full HPgTPC and the FV.]{Distributions of the true $\nu_{\mu}$ CC vertex positions for the full HPgTPC volume (blue) and the optimised FV (yellow), given by $\Delta L_{\mathrm{FV}} = 30.0 ~ \mathrm{cm}$ and $\Delta R_{\mathrm{FV}} = 50.0 ~ \mathrm{cm}$.}
    \label{fig:numuCC_true_vertex}
\end{figure}

\section{Charged pion identification}

\subsection[\texorpdfstring{$\nu_{\mu}$}{numu} CC \texorpdfstring{$1\pi^{\pm}$}{1pi} selection]{\boldmath\texorpdfstring{$\nu_{\mu}$}{numu} CC \boldmath\texorpdfstring{$1\pi^{\pm}$}{1pi} selection}


\section{Neutral pion identification}


\section{Systematic uncertainties}

\subsection{Flux uncertainties}

\subsection{Cross section uncertainties}

\subsection{Detector uncertainties}