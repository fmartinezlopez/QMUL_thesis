\chapter{Conclusion and outlook}
\label{chapter:conclusion}

\begin{chapquote}{Lucio Anneo Seneca, \textit{Epistulae morales ad Lucilium}}
    Our plans miscarry because they have no aim. When a man does not know what harbour he is making for, no wind is the right wind.
\end{chapquote}

% Executive summary
This thesis is a compilation of three different projects within DUNE. However, the idea behind each one of them is the same. The common theme is the prospect of improving or extending the physics of DUNE. In the first case, by enhancing the production of TPs in the induction channels what I seek is to provide more useful information to the FD data selection. The investigations with both data and MC, as well as the opportunity to run with a live detector, showed that such an enhancement is possible and should be pursued. Next, the solar DM analysis adds to the already rich BSM programme of DUNE. With the results of these preliminary studies, I want to show that DUNE can be complementary to the large-volume neutrino detectors in this kind of searches. Finally, the goal of the ND-GAr reconstruction improvements was the development of the PID strategy of the detector. For this, I tried to extract all the possible information from its different subcomponents. With the PID at hand, it is possible to form the selections of the different ND-GAr samples, as I have shown in this work. These will help understand how the detector is going to further constrain the neutrino interaction uncertainties in DUNE Phase II, which will eventually allow DUNE to reach its ultimate physics goals.

% Matched filter
% Results and lessons learned

% Next steps

% Solar Dark Matter
% Results and lessons learned

% Next steps

% ND-GAr reconstruction and selection
% Results and lessons learned

% Next steps

% Closing remarks