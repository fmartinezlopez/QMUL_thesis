\chapter{Conclusion and outlook}
\label{chapter:conclusion}

\begin{chapquote}{Lucio Anneo Seneca, \textit{Epistulae morales ad Lucilium}}
    Our plans miscarry because they have no aim. When a man does not know what harbour he is making for, no wind is the right wind.
\end{chapquote}

% Executive summary
This thesis is a compilation of three different projects within \gls{dune}. However, the idea behind each one of them is the same. The common theme is the prospect of improving or extending the physics of \gls{dune}. In the first case, by enhancing the production of trigger primitives in the induction channels what I seek is to provide more useful information to the far detector data selection. The investigations with both data and MC, as well as the opportunity to run with a live detector, showed that such an enhancement is possible and should be pursued. Next, the solar dark matter analysis adds to the already rich beyond the Standard Model physics programme of \gls{dune}. With the results of these preliminary studies, I want to show that \gls{dune} can be complementary to the large-volume neutrino detectors in these kinds of searches. Finally, the goal of the \gls{ndgar} reconstruction improvements was the development of the particle identification strategy of the detector. For this, I tried to extract all the possible information from its different subcomponents. With the particle identification at hand, it is possible to form the selections of the different \gls{ndgar} samples, as I have shown in this work. These will help understand how the detector is going to further constrain the neutrino interaction uncertainties in \gls{dune} Phase II, which will eventually allow \gls{dune} to reach its ultimate physics goals.

\begin{comment}
% Matched filter
% Results and lessons learned
The \gls{daq} system of the \gls{dune} far detector relies on the online identification of hits on channels, the so-called trigger primitives, to form decisions data to store. The goal of Chapter \ref{chapter:matched_filter} is to motivate a method to enhance the production of trigger primitives in the induction channels of the detectors. Forming trigger primitives from all the charge readout planes will improve the redundancy of the trigger algorithms. Not only that, but this may be the key to have more complex trigger logic that requires directional information. The aspect I focused on to improve the hit finding is the filtering of the waveforms. In section \ref{sec:matched_filter_fir} I use a sample of ProtoDUNE-SP cosmic data to show how different low-pass FIR filters affect the S/N in the collection and induction planes. Then, I introduce the concept of the matched filter in section \ref{sec:matched_filter_matched_filter}. Using the same dataset, I demonstrate that the improvement in the S/N of the induction channels achieved with these filters can be significantly higher than with the standard filter approach. A series of studies using MC samples are presented in section \ref{sec:matched_filter_mc_studies}. These allow to study the dependence of the filtering on the orientation and the energy of the tracks. I also use them to assess the impact of this method on the hit sensitivity. Finally, in section \ref{sec:matched_filter_vdcoldbox} I briefly summarise the results from the VD ColdBox runs which featured the matched filter.

% Next steps
With these studies, I showed that the matched filter puts the production of trigger primitives in the induction and collection planes on the same level. The natural next step will be to understand the impact that this has in the context of the current trigger algorithms. Then, explore the development of new trigger routines, like triggers based on coincidence across planes. At the same time, these alternative hit finder chains should be implemented in the trigger simulations currently under development.
\end{comment}

In Chapter \ref{chapter:matched_filter} I showed that the matched filter puts the production of trigger primitives in the induction and collection planes on the same level. The natural next step will be to understand the impact that this has in the context of the current trigger algorithms. Then, one could explore the development of new trigger routines, like triggers based on coincidence across planes. At the same time, these alternative hit finder chains should be implemented in the trigger simulations currently under development.

\begin{comment}
% Solar Dark Matter
% Results and lessons learned
The solar dark matter analysis is covered in Chapter \ref{chapter:dm_analysis}. There I explain how the \gls{dune} far detector can be used to probe dark matter interactions by measuring the neutrino flux coming from dark matter annihilations in the core of the Sun. After introducing the topic of dark matter capture and annihilation in a massive object like the Sun, I describe what kind of neutrino signals one can expect from such events in section \ref{sec:dm_analysis_flux}. Later, I comment on how \gls{dune} could constrain the dark matter parameter space by performing counting experiments. In section \ref{sec:dm_analysis_high_e_nu} I study the selection efficiency for the $\tau^{+}\tau^{-}$ and $b\bar{b}$ channels. I focus on two different kinematic regimes: the high energy neutrinos where \gls{dis} interactions with argon dominate, and the low energy part of the spectrum where neutrinos mainly undergo QEL interactions. This allows me to compute the projected generator-level dark matter cross section sensitivities, showing how \gls{dune} can be complementary to other indirect dark matter searches. Additionally, I explore two specific realisations of the dark matter interactions, namely Kaluza-Klein and leptophilic dark matter.

% Next steps
At this stage, this analysis already shows the potential of \gls{dune} to explore these scenarios. However, including the full simulation and reconstruction of the events will be necessary moving forward. At the moment, a significant effort is aimed towards the reconstruction of atmospheric neutrinos in the \gls{dune} far detector, which could be relevant for the case at hand. Also, following iterations of the analysis should include all the relevant systematic uncertainties. A summary of these is presented in section \ref{sec:dm_analysis_systematics}.
\end{comment}

At this stage, the solar dark matter analysis presented in Chapter \ref{chapter:dm_analysis} already shows the potential of \gls{dune} to explore these scenarios. However, including the full simulation and reconstruction of the events will be necessary moving forward. At the moment, a significant effort is aimed towards the reconstruction of atmospheric neutrinos in the \gls{dune} far detector, which could be relevant for the case at hand. Also, following iterations of the analysis should include all the relevant systematic uncertainties, summarised in section \ref{sec:dm_analysis_systematics}.

\begin{comment}
% ND-GAr reconstruction
% Results and lessons learned
Chapter \ref{chapter:garsoft_pid} reviews my work on the reconstruction for \gls{ndgar}. In section \ref{section:dEdx} I try to establish the relation between the measured charge in the readout and the deposited energy from a stopping proton sample, using the residual range of the tracks. This calibration allows to compute the mean $\mathrm{d}E/\mathrm{d}x$ for the particles. I finish the section providing a parametrisation for how this depends on the momentum. The problem of the muon and pion separation is the topic of section \ref{section:muon_bdt}. I propose to use the information from the \gls{ecal} to achieve this classification. In this section, I describe the features and the procedure I follow to train the classifier, showing its performance as a function of the particle momentum. In section \ref{section:tof} I explore the possibility of performing a \gls{tof} measurement with the \gls{ecal}. With this, I achieve a separation between pions and protons in a momentum range beyond the reach of the HPgTPC alone. Section \ref{section:pi_decay} is devoted to the identification of charged particle decays inside the HPgTPC where the parent plus (charged) daughter system is reconstructed as a single track. I use the information from the track fit to construct a series of variables which can identify the tracks containing decays, as well as locate their position. I finish the Chapter introducing a new clustering algorithm for the \gls{ecal} hits in section \ref{section:neutral}. It aims at having a one-to-one correspondence between particles and clusters, which will facilitate the reconstruction of neutral particles in the \gls{ecal}.

% Next steps
The goal of these developments was establishing a robust particle identification strategy for \gls{ndgar}, that allows to reconstruct the multiplicity of pions and other hadrons in the neutrino interactions final states. In section \ref{section:integration} I describe the status of the integration of the different additions to the reconstruction chain.
\end{comment}

The goal of the reconstruction developments discussed in Chapter \ref{chapter:garsoft_pid} was establishing a robust particle identification strategy for \gls{ndgar}, that allows to reconstruct the multiplicity of pions and other hadrons in the neutrino interactions final states. Following the integration efforts described in section \ref{section:integration}, the next steps include continue developing other aspects of the reconstruction.

\begin{comment}
% ND-GAr selection
% Results and lessons learned
Finally, in Chapter \ref{chapter:gar_selection} I apply to the event selection in \gls{ndgar}. I start by describing a method for selecting $\nu_{\mu}$ \gls{cc} events in section \ref{sec:gar_numu_cc}. This is mainly based on the muon score derived from the muon/pion classification I developed. Additionally, I perform an optimisation of the FV. As part of this study, I also examined the kinematics of the selected primary muon and the reconstructed interaction vertex. Next, in section \ref{sec:gar_charged_pions} I explore the capabilities of \gls{ndgar} and its reconstruction at identifying charged pions. I optimise a selection based on the reconstructed charged pion multiplicity, for events with 0, 1, 2, and $\geq 3 \pi^{\pm}$ in the final state. I the performance of the selection as a function of the truth hadronic invariant mass, as well as the true pion kinematics for the $\nu_{\mu}$ \gls{cc} $1\pi^{\pm}$ case. I briefly discuss the possibility of tagging events with neutral pions by reconstructing the invariant mass of the photon pairs from their decay in section \ref{sec:gar_neutral_pions}. Lastly, in section \ref{sec:gar_energy} I study the neutrino energy reconstruction of the selected $\nu_{\mu}$ \gls{cc} events using a calorimetric approach. For this, I compare the values obtained using generator-level and reconstructed information.

% Next steps
These studies constitute the first try at an event selection in \gls{ndgar} using full simulation and reconstruction. It will serve as a stepping stone for the development of other selections and analyses. Ultimately, the goal is to quantify the impact of \gls{ndgar} on the long baseline analysis in \gls{dune}. For this, including the effect of the systematic uncertainties outlined in section \ref{sec:gar_systematics} will be necessary.
\end{comment}

The studies in Chapter \ref{chapter:gar_selection} constitute the first try at an event selection in \gls{ndgar} using the end-to-end simulation and reconstruction. It will serve as a stepping stone for the development of other selections and analyses. Ultimately, the goal is to quantify the impact of \gls{ndgar} on the long baseline neutrino oscillation analysis in \gls{dune}. This will allow for a physics-driven optimisation of the detector design. For this, including the effect of the systematic uncertainties outlined in section \ref{sec:gar_systematics} will be necessary. Future work will also include the study of proton exclusive samples, neutral pion identification and neutron tagging, as well as comparisons with \gls{ndlar} + \gls{tms}. Additionally, these studies will go towards a publication on the capabilities of \gls{ndgar} and the additional physics potential that it will allow for \gls{dune}.

\begin{comment}
% Closing remarks
In summary, this thesis provides an overview of three novel topics within \gls{dune}. As a single sentence, in this work I investigate the enhancement of the triggering capabilities of the far detector, study the sensitivity of the far detector to solar dark matter signatures, and develop the particle identification and event selection strategies for the Phase II \gls{nd}. Each Chapter aims to be a comprehensive summary of the status of the different studies. I hope they can be helpful guides for future work both in the \gls{nd} and far detector.
\end{comment}