\chapter{DUNE Near Detector}
\label{chapter:dune_nd}

Before arriving to the FD, the neutrino beam meets the near detector (ND) complex, which serves as the experiment's control. The role of it is to measure the unoscillated neutrino energy spectra. From these we can predict the unoscillated spectra at the FD, which can be compared to the spectra measured at the FD in order to extract the oscillation parameters.

The design of the DUNE ND is mainly driven by the needs of the oscillation physics program. The capabilities of the ND have a direct impact on the experiment's sensitivity to the CP violating phase, mass ordering and other tests of the three flavours model.

In this section we aim to  

\section{Need for a ND}

One of the main roles of the ND is to measure the neutrino interaction rates before the oscillation effects become relevant, i.e. close to the production point. By measuring the $\nu_{\mu}$ and $\nu_{e}$ energy spectra and that of their corresponding antineutrinos, we can predict the unoscillated FD spectra using the ND measurements. As the neutrino spectra measured at the FD depends on the oscillation parameters, we are able to extract these via a fit using the no oscillation hypothesis.

To achieve the required precision for DUNE we need to minimise the systematic uncertainties affecting the observed neutrino energy. The reconstructed energy spectrum arises from a convolution of the flux, cross section and detector response. As we chose the LArTPC technology for the FD modules, the only way the ND can independently constrain all these kinds of systematic uncertainties is by featuring another LArTPC. This component of the ND complex, known as ND-LAr, uses fundamentally the same detection principle of the FD.

Another requirement for the ND is that it must reduce the impact of the cross section uncertainties on the measurement of the neutrino spectra. In other words, it needs to measure neutrino interactions much more accurately than the FD. This requires a better particle identification and energy reconstruction capabilities.

Additionally, the ND will have a physics program of its own. In particular, it will measure neutrino cross sections that will then be used to constrain the model used in the long-baseline oscillation analysis. It will also be used to search for BSM phenomena such as heavy neutral leptons, dark photons, millicharged particles, etc.

\section{ND overview}

\section{ND-GAr}

ND-GAr is a magnetised, high-pressure gaseous argon TPC (HPgTPC), surrounded by an electromagnetic calorimeter (ECal) and a muon detector (commonly refer to as $\mu$ID) (cite ND CDR).

In DUNE Phase II it will fulfill the role of TMS, measuring the momentum and sign of the charged particles exiting ND-LAr. Additionally, it will be able to measure neutrino interactions inside the HPgTPC, achieving lower energy thresholds than those of the ND and FD LArTPCs. By doing so ND-GAr will allow to constrain the relevant systematic uncertainties for the LBL analysis even further.

\subsection{Requirements}

The primary requirement for ND-GAr is to the measure the momentum and charge of muons from $\nu_{\mu}$ and $\bar{\nu}_{\mu}$ CC interactions in ND-LAr, in order to measure their energy spectrum. To achieve the sensitivity to the neutrino oscillation parameters described in the DUNE FD technical design report (TDR) (cite FD TDR vol 2) ND-GAr should be able to constrain the muon energy within a $1\%$ uncertainty or better. The main 

\subsection{Reference design}

\subsection{Expected performance}